% Options for packages loaded elsewhere
\PassOptionsToPackage{unicode}{hyperref}
\PassOptionsToPackage{hyphens}{url}
%
\documentclass[
]{article}
\usepackage{lmodern}
\usepackage{amssymb,amsmath}
\usepackage{ifxetex,ifluatex}
\ifnum 0\ifxetex 1\fi\ifluatex 1\fi=0 % if pdftex
  \usepackage[T1]{fontenc}
  \usepackage[utf8]{inputenc}
  \usepackage{textcomp} % provide euro and other symbols
\else % if luatex or xetex
  \usepackage{unicode-math}
  \defaultfontfeatures{Scale=MatchLowercase}
  \defaultfontfeatures[\rmfamily]{Ligatures=TeX,Scale=1}
\fi
% Use upquote if available, for straight quotes in verbatim environments
\IfFileExists{upquote.sty}{\usepackage{upquote}}{}
\IfFileExists{microtype.sty}{% use microtype if available
  \usepackage[]{microtype}
  \UseMicrotypeSet[protrusion]{basicmath} % disable protrusion for tt fonts
}{}
\makeatletter
\@ifundefined{KOMAClassName}{% if non-KOMA class
  \IfFileExists{parskip.sty}{%
    \usepackage{parskip}
  }{% else
    \setlength{\parindent}{0pt}
    \setlength{\parskip}{6pt plus 2pt minus 1pt}}
}{% if KOMA class
  \KOMAoptions{parskip=half}}
\makeatother
\usepackage{xcolor}
\IfFileExists{xurl.sty}{\usepackage{xurl}}{} % add URL line breaks if available
\IfFileExists{bookmark.sty}{\usepackage{bookmark}}{\usepackage{hyperref}}
\hypersetup{
  pdftitle={Precedentes vinculantes en materia laboral},
  pdfauthor={Paul Paredes},
  hidelinks,
  pdfcreator={LaTeX via pandoc}}
\urlstyle{same} % disable monospaced font for URLs
\setlength{\emergencystretch}{3em} % prevent overfull lines
\providecommand{\tightlist}{%
  \setlength{\itemsep}{0pt}\setlength{\parskip}{0pt}}
\setcounter{secnumdepth}{-\maxdimen} % remove section numbering

\title{Precedentes vinculantes en materia laboral}
\author{Paul Paredes}
\date{14 de agosto de 2021}

\begin{document}
\maketitle

\texttt{\{r\ xaringan-logo,\ echo=FALSE\}\ xaringanExtra::use\_logo(\ \ \ image\_url\ =\ "https://raw.githubusercontent.com/PaulParedes/materiales/master/logo-amag-removebg.png"\ )}

class: animated, fadeIn

\section{La responsabilidad (civil) en el ámbito laboral}

\subsection{1. Enfoque clásico}

\subsection{2. Enfoque desde la justicia distributiva y la justicia
correctiva}

class: animated, slideInRight

\section{2. El enfoque desde la justicia distributiva y la justicia
correctiva}

\begin{itemize}
\item
  Tendencia a la unificación de los sistemas de responsabilidad
\item
  Se habla de responsabilidad civil vs.~responsabilidad penal, pero
  abarca muchos campos, incluido el laboral
\item
  La unificación significa que los linderos de la responsabilidad
  contractual vs.~la extracontractual se difuminan
\item
  No se reconoce que hay diferencia entre que haya, o no, un contrato de
  por medio, pero ya no es un criterio de separación entre ``sistemas''.
\item
  P. ej. el CCyC argentino de 2014
\item
  O ``Los principios de derecho europeo de la responsabilidad civil'',
  Viena 2005
\item
  Se habla de derecho de daños, \emph{tort law}
\end{itemize}

class: animated, fadeIn

\section{Aspectos problemáticos en los casos de responsabilidad por
accidente de trabajo y por enfermedad profesional}

¿Qué tipo de responsabilidad aplica?

\begin{quote}
Artículo 53. Indemnización por daños a la salud en el trabajo
\end{quote}

\begin{quote}
El incumplimiento del empleador del deber de prevención genera la
obligación de pagar las indemnizaciones a las víctimas, o a sus
derechohabientes, de los accidentes de trabajo y de las enfermedades
profesionales. En el caso en que producto de la vía inspectiva se haya
comprobado fehacientemente el daño al trabajador, el Ministerio de
Trabajo y Promoción del Empleo determina el pago de la indemnización
respectiva.
\end{quote}

\begin{quote}
Ley 29783, LSST
\end{quote}

class: animated, fadeIn

\section{Aspectos a considerar en un caso de responsabilidad (por
despido arbitrario, accidente de trabajo o enfermedad profesional)}

\begin{itemize}
\item
  Hechos pasados y hechos futuros: extensión del daño
\item
  Identificar al (o los) responsable(s)
\item
  Identificar la(s) obligación(es) incumplida(s) (acciones, omisiones)
\item
  Identificar causas (concurrentes) y efectos
\item
  Identificar el tipo de daño (patrimonial, extrapatrimonial)
\item
  Cuantificar los daños
\item
  La prueba de los elementos de la teoría del caso
\end{itemize}

class: animated, bounceInDown, center, middle, inverse

\section{La prueba de la relación causal}

class: animated, fadeIn

\section{Teorías de la causalidad}

\subsection{3. Teorías que afirman la existencia de una necesaria
conexión intrínseca entre causa y efecto.}

\begin{itemize}
\tightlist
\item
  Teorías singularistas, antes que regulares, donde la causalidad es un
  proceso físico.
\end{itemize}

\subsection{4. Teorías de la probabilidad (estadística)}

class: animated, fadeIn

\subsection{\texorpdfstring{La causalidad es una relación cuaternaria:
\textbf{c} en lugar de \_\_C*\_\_ causa \textbf{e} antes que
\_\_E*\_\_}{La causalidad es una relación cuaternaria: c en lugar de \_\_C*\_\_ causa e antes que \_\_E*\_\_}}

\begin{quote}
El choque del camión, pero no el sistema eléctrico del camión, causa las
lesiones en el transeunte pero no su muerte.
\end{quote}

class: animated, fadeIn

\section{Causalidad mental}

\subsection{Causalidad descendente (Benjamin Libet, Walter
Sinnott-Armstrong)}

Hay un sistema superior donde se producen los eventos mentales (estados
mentales: intenciones, experiencias); y un sistema inferior donde se
producen los eventos físicos (estados físicos del cerebro).

\begin{quote}
¿Los estados mentales son estados físicos?
\end{quote}

class: animated, fadeIn

\section{Causalidad mental}

La causa son conceptos abstractos (Estados mentales) que llevan a hacer
o decir algo

Afirmaciones de la forma A causa B es verdadera por una relación causal
entre A y B de cualquier modo como A y B sean descritos; pero solo
ciertas descripciones de A y B terminan siendo una explicación.

Entonces:

Las razones son causas porque las razones explicativas son una especie
de explicación causal. Pero cualquier explicación causal ordinaria tiene
sentido respecto de un evento si ese evento encaja en el cuadro de las
leyes uniformes de la naturaleza; por su parte, las razones explicativas
hacen sentido sobre un comportamiento si encaja como un acto de un
sujeto racional.

class: center, middle, inverse, animated, bounceInDown

\section{¡Muchas gracias!}

\end{document}
